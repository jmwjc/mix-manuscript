\section{Mixed and penalty formulations for nearly-incompressible elasticity problems}
\subsection{Penalty formulation}
Consider a body $\Omega \in \mathbb R^{n_d}$ with boundary $\Gamma$ in $n_d$-dimension, where the $\Gamma_t$ and $\Gamma_g$ denotes its natural boundary and essential boundary such that $\Gamma_t \cup \Gamma_g = \Gamma$, $\Gamma_t \cap \Gamma_g = \varnothing$. The corresponding governing equations are given by:
\begin{equation}\label{strong_penalty}
\begin{cases}
    \nabla \cdot \boldsymbol \sigma + \boldsymbol b = \boldsymbol 0 & \mathrm{in} \; \Omega \\
    \boldsymbol \sigma \cdot \boldsymbol n = \boldsymbol t & \mathrm{on} \; \Gamma_t \\
    \boldsymbol u = \boldsymbol g & \mathrm{on} \; \Gamma_g \\
\end{cases}
\end{equation}
in which $\boldsymbol \sigma$ denotes to stress tensor and, for isotropic linear elastic material, can be expressed by: 
\begin{equation}\label{stress_penalty}
\boldsymbol \sigma(\boldsymbol u) = 3\kappa \boldsymbol \varepsilon^v(\boldsymbol u) + 2\mu \boldsymbol \varepsilon^d(\boldsymbol u) 
\end{equation}
where $\boldsymbol \varepsilon^v$ and $\boldsymbol \varepsilon^d$ are the volumetric(dilatation) and deviatoric parts of strain tensor $\boldsymbol \varepsilon$, and these are evaluated by:
\begin{equation}
\boldsymbol \varepsilon^v(\boldsymbol u) =\frac{1}{3} \nabla \cdot \boldsymbol u \; \boldsymbol 1,  \quad
\boldsymbol \varepsilon^d(\boldsymbol u) =\frac{1}{2}(\boldsymbol u \nabla + \nabla \boldsymbol u) - \boldsymbol \varepsilon^v, \quad
\boldsymbol \varepsilon^v : \boldsymbol \varepsilon^d = 0
\end{equation}
where $\boldsymbol 1 = \delta_{ij} \boldsymbol e_i \otimes \boldsymbol e_j$ is second order identity tensor.
$\kappa$, $\mu$ are the bulk modulus and shear modulus, and they can be represented by Young's modulus $E$ and Poisson's ratio $\nu$:
\begin{equation}\label{modulus}
\kappa = \frac{E}{2(1-2\nu)}, \quad \mu = \frac{E}{2(1+\nu)}
\end{equation}
And $\boldsymbol b$ denotes to prescribed body force in $\Omega$. $\boldsymbol t$, $\boldsymbol g$ are prescribed traction and displacement on natural and essential boundaries respectively. 

In accordance with Galerkin formulation, the displacement denoted by $\boldsymbol u$ can be got by the following weak problem: 
Find $\boldsymbol u \in V$
\begin{equation}\label{weak_penalty}
% \textrm{find} \; \boldsymbol u \in V, \\
\int_\Omega 2\mu \delta \boldsymbol \varepsilon^d : \boldsymbol \varepsilon^d d\Omega +
\int_\Omega 3\kappa \delta \boldsymbol \varepsilon^v : \boldsymbol \varepsilon^v d\Omega =
\int_{\Gamma_t} \delta \boldsymbol u \cdot \boldsymbol t d\Gamma + \int_\Omega \delta \boldsymbol u \cdot \boldsymbol b d\Omega,  \quad
\forall \delta \boldsymbol u \in V
\end{equation}
where $V$ is the spaces defined by $V=\{\boldsymbol v \in H^1(\Omega)^2\;\vert\;\boldsymbol v = \boldsymbol g, \; \textrm{on} \; \Gamma_g\}$. $\delta \boldsymbol u$ is the virtual counterpart of $\boldsymbol u$, and $\delta \boldsymbol \varepsilon^v$ and $\delta \boldsymbol \varepsilon^d$ are the corresponding volumetric and deviatoric strain evaluated by $\delta \boldsymbol u$.

% TODO: check construct mesh 
In traditional finite element formulation, the entire domain $\Omega$ is discretized by a set of construct mesh with vertices $\{\boldsymbol x_I\}_{I=1}^{n_p}$ \cite{hughes2000}, where $n_p$ is the total number of vertices. Then, the displacement and its virtual counterpart can be approximated by the nodal coefficient and shape functions at $\boldsymbol x_I$'s, the approximated displacement and its virtual counterpart, namely $\boldsymbol u_h, \delta \boldsymbol u_h$ have the following form: 
\begin{equation}\label{u_h}
\boldsymbol u_h(\boldsymbol x) = \sum_{I=1}^{n_p} N_I(\boldsymbol x) \boldsymbol u_I, \quad
\delta \boldsymbol u_h(\boldsymbol x) = \sum_{I=1}^{n_p} N_I(\boldsymbol x) \delta \boldsymbol u_I
\end{equation}
where $N_I$ and $\boldsymbol u_I$ are the shape function and nodal coefficient tensor at node $\boldsymbol x_I$.
Introducing Eq. \eqref{u_h} to weak form of Eq. \eqref{weak_penalty} leads to the following Ritz-Galerkin problem:
Find $\boldsymbol u_h \in V_h$,
\begin{equation}\label{ritz_penalty}
\int_\Omega 2\mu \delta \boldsymbol \varepsilon^d_h : \boldsymbol \varepsilon^d_h d\Omega +
\int_\Omega 3\kappa \delta \boldsymbol \varepsilon^v_h : \boldsymbol \varepsilon^v_h d\Omega =
\int_{\Gamma_t} \delta \boldsymbol u_h \cdot \boldsymbol t d\Gamma + \int_\Omega \delta \boldsymbol u_h \cdot \boldsymbol b d\Omega,  \quad
\forall \delta \boldsymbol u_h \in V_h
\end{equation}
where the approximate spaces $V_h \subseteq V$,
\begin{equation}
V_h = \{\boldsymbol v_h \in (\mathrm{span}\{N_I\}_{I=1}^{n_u})^2 \vert \boldsymbol v^h = \boldsymbol g,\; \mathrm{on} \; \Gamma_g\}
\end{equation}

For the arbitrariness of $\delta \boldsymbol u_h$, the above equation can be reduced by elimination of $\delta \boldsymbol u_I$'s as the following discrete equilibrium equation:
\begin{equation}\label{equilibrium_penalty}
(\boldsymbol K^d + \boldsymbol K^v) \boldsymbol d^u = \boldsymbol f
\end{equation}
where $\boldsymbol K^v$ and $\boldsymbol K^d$ are the volumetric and deviatoric stiffness matrices, and their components has the following forms:
\begin{equation}\label{stiffness_vol}
    \boldsymbol K^v_{IJ}  = 3\kappa \bar{\boldsymbol K}^v_{IJ},  \quad 
    \bar{\boldsymbol K}^v_{IJ}= \int_{\Omega} \boldsymbol B^{vT}_I \boldsymbol B^v_J d\Omega
\end{equation}
\begin{equation}
    \boldsymbol K^d_{IJ} = 2\mu \bar{\boldsymbol K}^d_{IJ}, \quad
    \bar{\boldsymbol K}^d_{IJ}= \int_{\Omega} \boldsymbol B^{dT}_I \boldsymbol B^d_J d\Omega
\end{equation}
with
\begin{equation}
\boldsymbol B^v_I =
\begin{bmatrix}
    \frac{1}{3}N_{I,x} & \frac{1}{3}N_{I,y} & \frac{1}{3}N_{I,z} \\
    \frac{1}{3}N_{I,x} & \frac{1}{3}N_{I,y} & \frac{1}{3}N_{I,z} \\
    \frac{1}{3}N_{I,x} & \frac{1}{3}N_{I,y} & \frac{1}{3}N_{I,z} \\
    0 & 0 & 0 \\
    0 & 0 & 0 \\
    0 & 0 & 0 \\
\end{bmatrix}
\end{equation}
\begin{equation}
\boldsymbol B^d_I = 
\begin{bmatrix}
     \frac{2}{3}N_{I,x} & -\frac{1}{3}N_{I,y} & -\frac{1}{3}N_{I,z} \\
    -\frac{1}{3}N_{I,x} &  \frac{2}{3}N_{I,y} & -\frac{1}{3}N_{I,z} \\
    -\frac{1}{3}N_{I,x} & -\frac{1}{3}N_{I,y} &  \frac{2}{3}N_{I,z} \\
     \frac{1}{2}N_{I,y} &  \frac{1}{2}N_{I,x} & 0 \\
     \frac{1}{2}N_{I,z} & 0                   &  \frac{1}{2}N_{I,x} \\
    0                   &  \frac{1}{2}N_{I,z} &  \frac{1}{2}N_{I,y} \\
\end{bmatrix}
\end{equation}
and $\boldsymbol f$ is the force vector and its components can be expressed by:
\begin{equation}
\boldsymbol f_I = \int_{\Gamma_t} N_I \boldsymbol t d\Gamma + \int_{\Omega} N_I \boldsymbol b d\Omega
\end{equation}
$\boldsymbol d^u$ is the coefficient vector containning $\boldsymbol u_I$'s.

% TODO: improve the expression.
It can be observed from Eq. \eqref{modulus} that, for a nearly-incompressible material, i.e. $\nu \rightarrow 0.5$, $\kappa \rightarrow \infty$. As a result, the volumetric stiffness matrix $\boldsymbol K^v$ of \eqref{stiffness_vol} services as an enforcement like penalty method to enforce the volumetric deformation to be zero, $\nabla \cdot \boldsymbol u = 0$, while the bulking modulus $\kappa$ can be regarded as a penalty parameter.
Traditional finite element formulations suffer severe volumetric locking due to this enforcement, and this is so-called volumetric locking. 
To reduce the burden of volumetric locking, the reduced the integration points in volumetric stiffness matrix. 

\subsection{Mixed formulation}
Another approach to alleviate the volumetric locking is using the mixed-formulation. In this approach, the pressure is approximated by another way as follows:
\begin{equation}\label{stress_mix}
\boldsymbol \sigma(\boldsymbol u, p) = -p \boldsymbol 1 + 2\mu \boldsymbol \varepsilon^d(\boldsymbol u)
\end{equation}
The strong form for mixed-formulation can be rephrased as:
\begin{equation}\label{strong_mix}
\begin{cases}
    \nabla \cdot \boldsymbol \sigma + \boldsymbol b = \boldsymbol 0 & \mathrm{in} \; \Omega \\
    \frac{p}{\kappa} + \nabla \cdot \boldsymbol u = 0 & \mathrm{in} \; \Omega \\
    \boldsymbol \sigma \cdot \boldsymbol n = \boldsymbol t & \mathrm{on} \; \Gamma_t \\
    \boldsymbol u = \boldsymbol g & \mathrm{on} \; \Gamma_g \\
\end{cases}
\end{equation}
where $p\in Q$, $Q = \{q \in L^2(\Omega) \vert \int_{\Omega} q d\Omega = 0\}$.

In traditional mixed formulations, the pressure $p$ are discretized by different sets of controlled nodes, namely displacement nodes $\{\boldsymbol x_I\}_{I=1}^{n_d}$ and pressure nodes $\{\boldsymbol x_K\}_{K=1}^{n_p}$, where $n_d$ and $n_p$ are the total number of displacement nodes and pressure nodes. And then the approximate displacement denoted by $\boldsymbol u_h$ and approximate pressure denoted by $p_h$ can be expressed by  
\begin{equation}\label{p_h}
    p_h(\boldsymbol x) = \sum_{K=1}^{n_p} \Psi_K(\boldsymbol x) p_K, \quad
    \delta p_h(\boldsymbol x) = \sum_{K=1}^{n_p} \Psi_K(\boldsymbol x) \delta p_K
\end{equation}
where $p_K$'s are the coefficients. and $N_I^d$, $N_K^p$ are the corresponding shape functions.
the corresponding Ritz-Galerkin problem is that:
Find $\boldsymbol u_h \in V_h$, $p_h \in Q_h$
\begin{subequations}\label{ritz_mix}
\begin{alignat}{2}
\label{ritz_mix_1}
\int_\Omega 2\mu \delta \boldsymbol \varepsilon^d_h : \boldsymbol \varepsilon^d_h d\Omega +
\int_\Omega \nabla \cdot \delta \boldsymbol u_h p d\Omega &=
\int_{\Gamma_t} \delta \boldsymbol u_h \cdot \boldsymbol t d\Gamma + \int_\Omega \delta \boldsymbol u_h \cdot \boldsymbol b d\Omega,  \quad
&\forall \delta \boldsymbol u_h \in V_h \\
\label{ritz_mix_2}
\int_\Omega \delta p_h \nabla \cdot \boldsymbol u_h d\Omega - \int_\Omega \frac{1}{3\kappa} \delta p_h p_h d\Omega &= 0, &\forall \delta p_h \in Q_h
\end{alignat}
\end{subequations}
where $Q_h \subseteq Q$ are defined by:
\begin{equation}
Q_h = \{q_h \in \mathrm{span}\{\Psi_K\}_{K=1}^{n_p} \vert \int_{\Omega} q_h d\Omega = 0\}
\end{equation}
Thus, $\dim V_h = 2n_d$ and $\dim Q_h = n_p$.
Eq. \eqref{ritz_mix_2} implies that $p_h$ is an orthogenal projection of $\nabla \cdot \boldsymbol u $
\begin{equation}
\Vert \nabla \cdot \boldsymbol u_h - p_h \Vert = \inf_{q_h \in Q_h} \Vert \nabla \cdot \boldsymbol u - q_h \Vert
\end{equation}
$p_h \in Q_h(V_h)$, $P(V_h) \subset V_h$.
\begin{equation}
    P_h(V_h) = \{\boldsymbol v_h \in V_h \vert \; \exists p_h \in P_h, \; \int_\Omega q_h(\nabla \cdot \boldsymbol v_h - p_h) d\Omega = 0, \; \forall q_h \in P_h \}
\end{equation}

With the arbitrariness of $\boldsymbol v_h$ and $q_h$, the Eq.\eqref{approxweak} leads to the following discrete governing equations:
\begin{equation}\label{equilibrium_mix}
    \begin{bmatrix}
        \boldsymbol K^{uu} & \boldsymbol K^{up} \\ (\boldsymbol K^{up})^T & -\boldsymbol K^{pp}
    \end{bmatrix}
    \begin{Bmatrix}
        \boldsymbol d^u \\ \boldsymbol d^p 
    \end{Bmatrix} =
    \begin{Bmatrix}
        \boldsymbol f \\ \boldsymbol 0 
    \end{Bmatrix}
\end{equation}
where $\boldsymbol K^{uu} = \boldsymbol K^d$. The components of $\boldsymbol K^{up}$ and $\boldsymbol K^{pp}$ yield:
\begin{align}
    \boldsymbol K^{up}_{IK} &= \int_{\Omega} 
    \begin{Bmatrix}
    N_{I,x} \\ N_{I,y}
    \end{Bmatrix} \Psi_K d\Omega \\
    K^{pp}_{KL} &= \int_\Omega\frac{1}{3\kappa} N_K N_L d\Omega
\end{align}

From the second equation of governing equilibrium equations in Eq. \eqref{equilibrium_mix}, the coefficient vector $\boldsymbol d^p$ can be expressed by $\boldsymbol d^u$ as follows:
\begin{equation}
    \boldsymbol d^p = (\boldsymbol K^{pp})^{-1} (\boldsymbol K^{up})^T \boldsymbol d^u
\end{equation}
Further substituting the above equation into first equation of Eq. \eqref{equilibrium_mix} leads to:
\begin{equation}
    (\underbrace{\boldsymbol K^{uu}}_{\boldsymbol K^d} + \underbrace{\boldsymbol K^{up}(\boldsymbol K^{pp})^{-1}(\boldsymbol K^{up})^{T}}_{3\kappa \tilde{\boldsymbol K}^v}) \boldsymbol d^u = \boldsymbol f
\end{equation}

% TODO: orthogenal projection

\begin{equation}
\begin{split}
    \boldsymbol K^v_{IJ} &= \sum_{K,L=1}^{n_p}\boldsymbol K^{up}_{IK} K^{pp-1}_{KL} \boldsymbol K^{upT}_{LJ} \\
    &= \sum_{K,L=1}^{n_p}\boldsymbol K^{up}_{IK} K^{pp-1}_{KL} \boldsymbol K^{upT}_{LJ} \\
    &= \sum_{K,L,M,N=1}^{n_p} \boldsymbol K^{up}_{IK} K^{pp-1}_{KM}  K^{pp}_{MN} K^{pp-1}_{NL} \boldsymbol K^{upT}_{LJ} \\
    &= \int_\Omega (\sum_{K,M=1}^{n_p} \Psi_M K^{pp-1}_{KM} \boldsymbol K^{up}_{IK}) (\sum_{L,N=1}^{n_p}\Psi_N K^{pp-1}_{NL} \boldsymbol K^{upT}_{LJ}) d\Omega \\
\end{split}
\end{equation}

\begin{equation}
\begin{split}
    \boldsymbol K^v &= \boldsymbol K^{up} \boldsymbol K^{pp-1} \boldsymbol K^{upT} \\
    &= \boldsymbol K^{up} \boldsymbol K^{pp-1} \boldsymbol K^{pp} (\boldsymbol K^{pp-1} \boldsymbol K^{up})^T \\
\end{split}
\end{equation}
