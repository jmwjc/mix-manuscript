\section{Optimal polynomial--wise constraint count}
\subsection{Inf--sup value estimator}
% Consider a general mixed-formulation problem, namely, find $u\in V, p\in Q$,
% \begin{equation}
% \begin{aligned}
%     a(v, u) + b(v, p) &= f(v) \quad &\forall v \in V \\
%     b(u, q) &= g(q) \quad &\forall q \in Q
% \end{aligned}
% \end{equation}
% where 
Without loss generality, the nearly-incompressible elasticity problem is considered herein to illustrate the proposed methodology.
The approximations of Eq.\eqref{approx} should satisfy the so-call Ladyzhenskaya–Babuška–Brezzi(LBB) condition or inf-sup condition \cite{bathe1996} to ensure the formulation's accuracy:
\begin{equation}\label{infsup}
    \inf_{q_h \in Q_h} \sup_{\boldsymbol v_h \in V_h} \frac{\vert b(q_h,\boldsymbol v_h) \vert}{\Vert q_h \Vert_Q \Vert \boldsymbol v_h \Vert_V} \ge \beta > 0
\end{equation}
in which $\beta$, namely inf-sup value, is a constant independent of characterized element size $h$.

\begin{thm}\label{thm1}
    Suppose $\mathcal P_h:V_h \rightarrow Q_h$ is the orthogonal projection operator of $\mathcal P$ defined by:
\begin{equation}
    b(q_h,\boldsymbol v_h) = (q_h, \mathcal P \boldsymbol v_h) = (q_h, \mathcal P_h \boldsymbol v_h), \quad \forall q_h \in Q_h
\end{equation}
    where $\mathcal P$ is the divergence operator, $\mathcal P = \nabla \cdot$. And the inf-sup value can be estimated by:
\begin{equation}\label{r1}
    \beta \le \inf_{V'_h \subset V_h \setminus \ker \mathcal P_h} \sup_{v_h \in V'_h} \frac{\Vert \mathcal P_h \boldsymbol v_h \Vert_Q}{\Vert \boldsymbol v_h \Vert_V}
\end{equation}
\end{thm}
\begin{pf}
    As the definition of  $\mathcal P_h$, we have $\mathrm{Im}\mathcal P_h \in Q_h$. And the Eq. \eqref{infsup} can be rewritten as:
\begin{equation} \label{r11}
\begin{split}
    \beta &\le \inf_{q_h \in Q_h} \sup_{\boldsymbol v_h \in V_h} \frac{\vert b(q_h,\boldsymbol v_h) \vert}{\Vert q_h \Vert_Q \Vert \boldsymbol v_h \Vert_V} 
    \le \inf_{q_h \in \mathrm{Im}\mathcal P_h} \sup_{\boldsymbol v_h \in V_h} \frac{\vert (q_h,\mathcal P_h \boldsymbol v_h) \vert}{\Vert q_h \Vert_Q \Vert \boldsymbol v_h \Vert_V} \\
\end{split}
\end{equation}

    For a given $q_h\in \mathrm{Im}\mathcal P_h$, suppose a space $V'_h \subseteq V_h\setminus \ker P_h$ defined by:
\begin{equation}
    V'_h = \{ \boldsymbol v_h \in V_h \; \vert \; \mathcal P_h \boldsymbol v_h = q_h \}
\end{equation}
    Since $\mathrm{Im}\mathcal P_h \in Q_h$, in accordance with Cauchy-Schwarz inequality, we have:
\begin{equation}
    \vert (q_h,\mathcal P_h \boldsymbol v_h) \vert \le \Vert q_h \Vert_Q \Vert \mathcal P_h \boldsymbol v_h \Vert_Q
\end{equation}
where this equality is holding if and only if $q_h=\mathcal P_h \boldsymbol v_h$, i.e.,
\begin{equation}
    \vert (q_h,\mathcal P_h \boldsymbol v_h) \vert = \Vert q_h \Vert_Q \Vert \mathcal P_h \boldsymbol v_h \Vert_Q, \quad \forall \boldsymbol v_h \in V'_h
\end{equation}
And the following relationship can be evidenced:
    \begin{equation}\label{r12}
    \sup_{\boldsymbol v_h\in V_h} \frac{\vert (q_h,\mathcal P_h \boldsymbol v_h) \vert}{\Vert q_h \Vert_Q \Vert \boldsymbol v_h \Vert_V} =
    \sup_{\boldsymbol v_h\in V'_h} \frac{\Vert \mathcal P_h \boldsymbol v_h \Vert_Q}{\Vert \boldsymbol v_h \Vert_V} 
\end{equation}

    Consequently, with a combination of Eqs. \eqref{r11} and \eqref{r12}, Eq. \eqref{r1} can be obtain.
\end{pf}

\begin{rmk}
    Theorem \ref{thm1} is consistance with the traditional numerical inf-sup test \cite{malkus1981}, while, according to minimum-maximum principle \cite{babuskaa}, the Eq. \eqref{r1} evaluates the general non-zero eigenvalue of metrics $\boldsymbol K^v$ and $\boldsymbol K^d$.
\end{rmk}

In order to further figure out the optimal constraint counting ,
\begin{thm}
    Suppose that $P_{n_u}$ is a polynomial space with $n_u$ dimensions, and $V_{n_u}$ is the polynomial displacement space, $V_{n_u} = \{\boldsymbol v \in P_{n_u}^2 \vert \boldsymbol v=\boldsymbol g,\;\mathrm{on}\; \Gamma_g\}$. And then, the inf-sup can be further bounded by:
\begin{equation}\label{r2}
    \beta \le C \inf_{V'\in V_{n_u}\setminus\ker \mathcal P}\sup_{\boldsymbol v \in V'}\frac{\Vert \mathcal P \boldsymbol v\Vert_Q}{\Vert\boldsymbol v\Vert_V} + Ch^{\min\{k,l\}}
\end{equation}
where $C$ stands for an arbitrary constant.
\end{thm}
\begin{pf}
    As the dimensions of $V_h$ and $V_{n_u}$ is identical, $\dim V_{n_u}=\dim V_h = n_d\times n_u$. There exists a unique $\boldsymbol v \in V_{n_u}$ satisfing $v_h = \mathcal I_h v$. And the right side of Eq. \eqref{r1} becomes:
\begin{equation}\label{r21}
\inf_{V'_h \in V_h\setminus \ker \mathcal P_h}\sup_{\boldsymbol v_h \in V_h'} \frac{\Vert \mathcal P_h \boldsymbol v_h \Vert_Q}{\Vert \boldsymbol v_h \Vert_V} = 
\inf_{V'\in V_{n_u}\setminus \ker \mathcal P_h \mathcal I_h}\sup_{\boldsymbol v \in V'} \frac{\Vert \mathcal P_h \mathcal I_h \boldsymbol v \Vert_Q}{\Vert \mathcal I_h \boldsymbol v \Vert_V}
\end{equation}
Since $\ker \mathcal P_h \mathcal I_h \subseteq \ker \mathcal P$, the above equation can be rewritten as:
\begin{equation}\label{r22}
\inf_{V'_h \in V_h\setminus \ker \mathcal P_h}\sup_{\boldsymbol v_h \in V_h'} \frac{\Vert \mathcal P_h \boldsymbol v_h \Vert_Q}{\Vert \boldsymbol v_h \Vert_V} \le 
\inf_{V'\in V_{n_u}\setminus \ker \mathcal P}\sup_{\boldsymbol v \in V'} \frac{\Vert \mathcal P_h \mathcal I_h \boldsymbol v \Vert_Q}{\Vert \mathcal I_h \boldsymbol v \Vert_V}
\end{equation}
In accordance with triangular inequality, we have:
\begin{equation}\label{interpolation1}
\begin{split}
    \Vert \mathcal P_h \mathcal I_h \boldsymbol v \Vert_Q &\le \Vert \mathcal P \boldsymbol v \Vert_Q + \Vert \mathcal P \boldsymbol v - \mathcal P_h \mathcal I_h \boldsymbol v \Vert_Q \\
    &\le \Vert \mathcal P \boldsymbol v \Vert_Q + \Vert \mathcal P \boldsymbol v - \mathcal P \mathcal I_h \boldsymbol v \Vert_Q + \Vert \mathcal P \mathcal I_h \boldsymbol v - \mathcal P_h \mathcal I_h \boldsymbol v \Vert_Q \\
    &\le \Vert \mathcal P \boldsymbol v \Vert_Q + \Vert \mathcal P(\mathcal I - \mathcal I_h) \boldsymbol v \Vert_Q + \Vert (\mathcal P - \mathcal P_h) \mathcal I_h \boldsymbol v \Vert_Q \\
\end{split}
\end{equation}
Obviously, the second and third terms on the right side of Eq. \eqref{interpolation1} are the interpolation error and the orthogonal projection error for approximations in $V_h$ and $Q_h$, and can be evaluated by \cite{yosida1995}:
\begin{align}
    &\Vert \mathcal P(\mathcal I - \mathcal I_h) \boldsymbol v \Vert_Q \le C h^k \Vert \boldsymbol v \Vert_V \\
    &\Vert(\mathcal P - \mathcal P_h)\mathcal I_h \boldsymbol v \Vert_Q \le Ch^l \Vert \mathcal I_h \boldsymbol v \Vert_V
\label{interpolation2}
\end{align}
    It can be obtained that $\Vert \mathcal I_h \boldsymbol v\Vert_V \ge C\Vert \boldsymbol v \Vert_V$ from close graph theorem \cite{quarteroni1994}. And considering it with Eqs. \eqref{interpolation1}-\eqref{interpolation2}, the right side of Eq. \eqref{r22} can be represented as:
\begin{equation}\label{r23}
\inf_{V'\in V_{n_u}\setminus \ker \mathcal P} \sup_{\boldsymbol v \in V'} \frac{\Vert \mathcal P_h\mathcal I_h\boldsymbol v\Vert_Q}{\Vert \mathcal I_h \boldsymbol v\Vert_V} \le
    C \inf_{V'\in V_{n_u}\setminus \ker \mathcal P} \sup_{\boldsymbol v \in V'} \frac{\Vert \mathcal P\boldsymbol v\Vert_Q}{\Vert \boldsymbol v\Vert_V} + Ch^{\min\{k,l\}}
\end{equation}

    Consequently, substituting Eqs. \eqref{r21},\eqref{r22},\eqref{r23} into \eqref{r1} can get the desired Eq. \eqref{r2}.
\end{pf}

\subsection{Optimal constraint count}
\begin{table}[ht!]
\centering
\caption{Degrees of freedom and volumetric constraint}
\begin{tabular}{ccc}
\toprule
$p$ & $2n_u$ & $c$ \\
\midrule
1 & 6 & 5 \\
2 & 12 & 9 \\
3 & 20 & 14 \\
4 & 30 & 20 \\
$p$ & $\sum_{i=1}^{p+1}2i$ & $\sum_{i=1}^{p+1}i+1$ \\
\bottomrule
\end{tabular}
\end{table}
\subsection{Equivalence with inf-sup test}
