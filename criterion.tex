\section{Optimal polynomial-wise constraint count}
\subsection{Inf-sup value estimator}
The problem of Eqs.\eqref{approxweak} the approximations of Eq.\eqref{approx} should satisfy the so-call Ladyzhenskaya–Babuška–Brezzi(LBB) condition or inf-sup condition \cite{bathe1996} to ensure the formulation's accuracy:
\begin{equation}
    \inf_{q_h \in Q_h} \sup_{\boldsymbol v_h \in V_h} \frac{\int_{\Omega} q_h \nabla \cdot \boldsymbol v_h d\Omega}{\Vert q_h \Vert_{L^2} \Vert \boldsymbol v_h \Vert_{H^1}} = \beta_h \ge \beta > 0
\end{equation}
in which $\beta_h$ is namely inf-sup value, $\beta$ stands for a constant independent of characterized element size $h$.

\begin{thm}
% FIXME: find a ref for polynomial interpolation.
    Let the polynomial space of degree at most $k$, $\mathcal L_{(k)}$ is defined as follows:
\begin{equation}
    \mathcal L_{(k)} := ??
\end{equation}
such that, the inf-sup value $\beta_h$
    Inf-sup value has the following relationship:
\begin{equation}
    \beta_h \le \sqrt{\lambda^{(n_d n_u-n_p+1)}} + C h \Vert \boldsymbol v \Vert
\end{equation}
with
\begin{equation}
    \lambda^{(k)} = \inf_{W \in \mathcal L_{(2n_u)}^{(k)}} \sup_{\boldsymbol v \in W} 
    \frac{\int_\Omega (\nabla \cdot \boldsymbol v)^2 d\Omega}{\Vert \boldsymbol v \Vert_{H^1}^2}
\end{equation}
\end{thm}
\begin{pf}

\end{pf}
assuming that, $\boldsymbol v$ is a polynomial in polynomial space $\mathcal L$ of degree at most $2n_u$, i.e. $\boldsymbol v \in \mathcal L_{(2n_u)}$. 
As $\mathcal L_{2n_u} \subset V$, $\boldsymbol v$ can satisfied with problem of Eq.\eqref{weak}.
In accordance with $Q_h \subseteq Q$, the following relationship hold true:
\begin{equation}\label{b2}
b(\boldsymbol v,q_h) = \int_{\Omega} q_h \nabla \cdot \boldsymbol v d\Omega = 0, \quad \forall q_h \in Q_h
\end{equation}
A direct subtraction between the second equation of Eq.\eqref{approxweak} and Eq.\eqref{b2} yields: 
\begin{equation}
b(\boldsymbol v - \boldsymbol v_h, q_h) = \int_{\Omega} q_h (\nabla\cdot \boldsymbol v - \nabla \cdot \boldsymbol v_h)d\Omega = 0, \quad \forall q_h \in Q_h
\end{equation}
The above relation means that $\nabla \cdot \boldsymbol v_h$ is the elliptic projection of $\nabla \cdot \boldsymbol v$ onto $Q_h$, 
$\nabla \cdot \boldsymbol v \in Q_h(V_h)$
To further development of this methodology, the inf-sup value $\beta_h$ can be firstly derived as follows:
\begin{equation}
\begin{split}
\beta_h &= \inf_{q_h \in Q_h} \sup_{\boldsymbol v_h \in V_h} \frac{\int_{\Omega} q_h \nabla \cdot \boldsymbol v_h d\Omega}{\Vert q_h \Vert_{L^2} \Vert \boldsymbol v_h \Vert_{H^1}} \\
    &= \inf_{\boldsymbol w_h \in \bar{Q}_h(V_h)} \sup_{\boldsymbol v_h \in \bar{Q}_h(V_h)} \frac{\int_{\Omega} \nabla \cdot \boldsymbol w_h \nabla \cdot \boldsymbol v_h d\Omega}{\Vert \boldsymbol w_h \Vert_{H(\textrm{div})} \Vert \boldsymbol v_h \Vert_{H^1}} \\ 
    &= \inf_{W \subset \bar{Q}_h(V_h)} \sup_{\boldsymbol v_h \in  W} \frac{\sqrt{ \int_{\Omega} (\nabla \cdot \boldsymbol v_h)^2 d\Omega}}{\Vert \boldsymbol v_h \Vert_{H^1}} \\
    &= \sqrt{\inf_{W \subset \bar{Q}_h(V_h)} \sup_{\boldsymbol v_h \in W} \frac{ \int_{\Omega} (\nabla \cdot \boldsymbol v_h)^2 d\Omega}{\Vert \boldsymbol v_h \Vert^2_{H^1}}} \\
\end{split}
\end{equation}

The space $\ker B_h$ and its counterpart $\ker  B_h^*$ defined by:
\begin{align}
    \ker B_h &= \{q_h \in Q_h \vert \int_{\Omega} q_h \nabla \cdot \boldsymbol v_h d\Omega = 0,\; \forall \boldsymbol v_h \in V_h\} \\
    \ker B_h^* &= \{\boldsymbol v_h \in V_h \vert \int_{\Omega} q_h \nabla \cdot \boldsymbol v_h d\Omega = 0,\; \forall q_h \in Q_h\}
\end{align} \par

% TODO: add the estimator of interpolation approximation

\begin{equation}
\begin{split}
    \lambda^{(k)}_h &= \inf_{W \subset Q_h^{(k)}(V_h)} \sup_{\boldsymbol v_h \in W} \frac{\int_{\Omega} (\nabla \cdot \boldsymbol v_h)^2 d\Omega}{\Vert \boldsymbol v_h \Vert^2} \\
    &\le \inf_{W \subset L^{(k)}_{(n_u)}(V)} \sup_{\boldsymbol v \in W} \frac{\int_{\Omega} (\nabla \cdot (\Pi_h \boldsymbol v))^2 d\Omega}{\Vert \Pi_h \boldsymbol v \Vert^2} \\
    &= \inf_{W \subset L^{(k)}_{(n_u)}(V)} \sup_{\boldsymbol v \in W} \frac{\int_{\Omega} (\nabla \cdot \boldsymbol v)^2 d\Omega}{\Vert \boldsymbol v \Vert^2}
    \frac{\Vert \boldsymbol v \Vert^2}{\Vert \Pi \boldsymbol v \Vert^2}\\
\end{split}
\end{equation}

\begin{equation}
\Vert \Pi_h \boldsymbol v - \boldsymbol v \Vert \le \Vert \Pi_h \boldsymbol v \Vert - \Vert \boldsymbol v \Vert \le Ch^{p+1}\Vert \boldsymbol v \Vert
\end{equation}

\begin{equation}
    \Vert \Pi_h \boldsymbol v \Vert \le (1+Ch^{p+1})\Vert \boldsymbol v \Vert
\end{equation}

\begin{equation}
    \Vert \boldsymbol v - \Pi_h \boldsymbol v \Vert_{H(div)} \le \Vert \boldsymbol v \Vert - \Vert \Pi_h \boldsymbol v \Vert \le Ch^{p}\Vert \boldsymbol v \Vert
\end{equation}

\begin{equation}
\frac{\Vert \boldsymbol v \Vert_1^2}{\Vert \Pi_h \boldsymbol v \Vert_1^2} \le \frac{1}{(1-Ch^{p+1})^2} \approx 1 + 2Ch^{p+1}
\end{equation}

\begin{equation}
    \mathcal L^{(n_u)}
\end{equation}
\begin{equation}
    \Vert \sum_{k=1}^{n_u} \mathcal L^{(k)} \boldsymbol v \Vert \le C h
\end{equation}
Interpolation error for polynomials, 

Triangular inequality
\begin{equation}
\Vert \Pi_h \boldsymbol v \Vert - \Vert \boldsymbol v \Vert \le \Vert \boldsymbol v - \Pi_h \boldsymbol v \Vert \le Ch^{p} \Vert \boldsymbol v \Vert
\end{equation}
\begin{equation}
\Vert \boldsymbol v \Vert - \Vert \Pi_h \boldsymbol v \Vert \le \Vert \boldsymbol v - \Pi_h \boldsymbol v \Vert \le Ch^{p} \Vert \boldsymbol v \Vert
\end{equation}

\subsection{Optimal constraint count}
\begin{table}[ht!]
\centering
\caption{Degrees of freedom and volumetric constraint}
\begin{tabular}{ccc}
\toprule
$p$ & $2n_u$ & $c$ \\
\midrule
1 & 6 & 5 \\
2 & 12 & 9 \\
3 & 20 & 14 \\
4 & 30 & 20 \\
p & $\sum_{i=1}^{p+1}2i$ & $\sum_{i=1}^{p+1}i+1$ \\
\bottomrule
\end{tabular}

\subsection{Equivalence with inf-sup test}
\end{table}
