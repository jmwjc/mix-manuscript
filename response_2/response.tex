\documentclass{article}
\usepackage{amsmath,amssymb,amsfonts,amsthm,bm}
\usepackage{float,caption,booktabs,multirow} % tables
\usepackage{enumerate}
\bibliographystyle{plain}

\RequirePackage[normalem]{ulem} 
\RequirePackage{color}\definecolor{RED}{rgb}{1,0,0}\definecolor{BLUE}{rgb}{0,0,1} 
\providecommand{\DIFaddtex}[1]{{\protect\color{blue}\uwave{#1}}} %DIF PREAMBLE
\providecommand{\DIFdeltex}[1]{{\protect\color{red}\sout{#1}}}                      %DIF PREAMBLE
\title{Response to Reviewer's Comments}
\author{}
\date{}
% \setlength{\parindent}{0em}
\setlength{\parskip}{1em}
\begin{document}

\maketitle
Authors are grateful to the reviewers for their meticulous review of our manuscript and for their constructive suggestions.
These comments have been instrumental in enhancing the paper's clarity and rigor.
We have thoroughly revised the manuscript in response to each point raised.
A detailed discussion of how we have addressed the reviewers' concerns follows.
All modifications are highlighted in the marked revised manuscript.

\section*{Reviewer \#1}
\textit{I find the reviewer responses to be quite "dismissive" to some of my comments and the comments of the other reviewer. Adding a small paragraph here and there will not cut it. Also, putting the table where computational costs grow tenfold and condition numbers increase by orders of magnitude and NOT including it, and the corresponding discussion, in the body of the paper is not acceptable. This was my exact issue: Is the method justified (beyond being a computational math curiosity)? Your new table suggests that you might as well forego FEM and just do Meshfree.}

\textbf{Response:} 
We sincerely apologize for the impression that our previous responses were dismissive.
That was not our intention, and we take your comments and the other reviewer's feedback very seriously.
While it is true that the proposed mixed FE--meshfree formulations show increased computational cost and condition numbers compared to traditional FEM, the primary purpose of this work is not to present a computationally superior method in all aspects.
Our main objective is to propose a novel and more precise optimal constraint ratio.
This ratio is theoretically grounded in the inf--sup condition, providing a robust framework to justify the satisfaction of this condition by simply counting the degrees of freedom of the displacement and pressure fields.
This optimal constraint ratio is not limited to our proposed formulation but is also applicable to other u--p mixed formulations.
The mixed FE--meshfree formulation serves as a flexible and effective tool to demonstrate the validity of this ratio.
Unlike element-based methods, the meshfree approximation allows for the independent placement of pressure DOFs, enabling us to test the constraint ratio with arbitrary values.
Additionally, this approach provides a general framework for constructing locking--free formulations, as the pressure meshfree approximation can be coupled with various displacement discretizations, such as isogeometric analysis.

We agree that the efficiency of the mixed FE--meshfree formulation is a critical issue that must be addressed.
As you suggested, we have added a comprehensive discussion on the condition number and efficiency.
The corresponding data is included in Table 6 for the Cook's membrane problem on Page 35 of the revised manuscript.
Moreover, a summary regarding efficiency and condition number has also been added in the conclusion section on the designated page.

\textit{I do appreciate the effort on additional computations. However, while illuminating, these are raising more questions than providing answers. For the cantilever beam, Cook's membrane, and compression block problems the following needs to be addressed:}

\textbf{Comment 1.} \textit{The authors are right. Since gradients of RK functions are not integrated, this makes the case simpler. However, I'm not convinced "lower-order Gauss quadrature works". Please specify precisely the Gauss quadrature used for each case. Make sure results are independent of Gauss quadrature. (You are still integrating a complicated function with a support that does not conform to the mesh…) Keep increasing the number of Gauss points per element to show that the results are not changing.}

\textbf{Response:} 
Authors thank the reviewer for their valuable suggestion.
We have now specified the precise Gauss quadrature schemes used in this work in Table 3 on Page 16 of the revised manuscript.
Specifically, methods with linear basis functions employ an integration scheme of order 2.
Methods with quadratic basis functions utilize a scheme of order 4.
It is a fact that lower-order integration rules cannot exactly integrate the meshfree shape functions.
However, this only affects the error and does not influence the accuracy, or convergence rate.
This is because it is not necessary to fulfill the integration constraint.
Consequently, solutions obtained with lower-order integration rules maintain a satisfactory level of accuracy.

Furthermore, a sensitivity study on the integration order has been added to Section 5.2 on Pages 29-30.
The results from this study demonstrate that the proposed mixed formulations are not sensitive to the integration order.

\textbf{Comment 2.} \textit{Why is the cantilever beam problem solved with Quads and plate with a hole with Tris only? Please show both element types on both problems. The Quads and Tris seem to behave very differently.}

\textbf{Response:}
Thanks for reviewer's suggestion. The studies of error convergence and sensitivity to $n_p$ for the cantilever beam problem using triangular elements and the plate with a hole problem using quadrilateral elements have been added in Section 5.1 and Section 5.2 of the revised manuscript, respectively, as shown in Figures.
The performances of triangular and quadrilateral elements exhibit almost the comparable,
expected two points: one is the strain errors respect to $n_p$ for Quad8--RK for cantilever problem; The other is also the strain errors for Tri3--RK and Quad8 for plate with hole problem. The two different point were also mentioned in reviewer's comments 3, 4. 
To further well explain these phenomenons, the strain and pressure error estimators for mixed--formulation have been added in Appendix on Page 45.
The added discussions on Pages 22, 29 imply that all the situations are under the context of error estimators.
The detailed discussions about these two points can be found in responses of comments 3 and 4 or in the revised manuscript.

\textbf{Comment 3.} \textit{Figure 12: Quadratic Quads error is completely insensitive to the number of constraints used. This makes no sense. In addition, as the number of constraints goes to zero, the errors in the strain and pressure do not increase. Why is that? Aren't you losing incompressibility completely in this limit? There must be some "pathology" with this problem, and this may not be a good example.}

\textbf{Response:} 
Thanks the reviewer to point out these interesting phenomenons.
First, you are correct that the strain error of the Quad8-RK method appears to be completely insensitive to the number of pressure nodes, $n_p$, in a uniform mesh. This is not a pathology of the problem but a direct consequence of the underlying error theory for mixed formulations.

To clarify this, we've added error estimators to the Appendix of the revised manuscript. As shown in Eq. (A.10), the strain error estimator is bounded by strain interpolation error under space $\ker \mathcal P_h$, pressure interpolation error and coercivity constant $\alpha$.
However, for a regular mesh, the Quad8 element satisfies a specific condition where its kernel space $\ker \mathcal P_h$ interpolation error is equivalent to the full space interpolation error. This makes the strain error independent of the inf-sup value $\beta$.
As a result, the strain error remains stable, regardless of changes in $n_p$.

To demonstrate that this is specific to uniform meshes, we have also conducted a test with a non-uniform discretization for both displacement and pressure fields.
As shown in the updated Figure 15 in the revised manuscript, the strain error of the Quad8-RK method is no longer stable; it increases as $n_p$ falls outside the optimal range. The related discussion has been added on Page 22, Lines 439-442.

Second, the observation that strain and pressure errors do not increase as $n_p$ approaches zero is also insightful. This behavior is linked to the specific solution of the problem. As detailed in Eqs. (A.10) and (A.14) in the revised manuscript, the error estimators are controlled by two terms: one dependent on displacement interpolation error and the other on pressure interpolation error.

In this particular problem, the exact pressure solution in Eq. (62) is only a second-order polynomial. As a result, the pressure interpolation error is significantly smaller than the displacement interpolation error, and in some cases, it can even be zero for quadratic approximations. Because the error estimators are also dependent on this small pressure interpolation error term, the overall strain and pressure errors do not immediately increase as the number of pressure nodes decreases. The related discussion has been added on Page 22, Lines 432-437.

\textbf{Comment 4.} \textit{
Figure 15:
The Tri3 results suggest that the estimate for $n_s$ is not very good. There is no sharp drop in the strain error around that value at all.
It looks like significantly increasing the number of incompressibility constraints does not result in higher errors.
Why is that? Shouldn't you be locking at some point? Tri6, however, seems to behave as designed. Very confusing and not substantiated by the analysis or other intuition/experience.
}

\textbf{Response:}
Authors appreciate the reviewer's observation of this interesting phenomenon.
Tri3--RK indeed does not show a sharp drop in the strain error when the $n_p$ goes below the optimal value $n_s$.
This phenomenon also appears in the additional test of strain error repect to $n_p$ for Tri6--RK in Figure 20 on Page 31.
This is because, as shown in Eqs. (A.10) and (A.20), the displacement approximation error for the space of $\ker \mathcal P_h$ does not increase as immediately when $\frac{C_b}{\beta}$ in Eq. (A.20) is not too much larger than 1.
The related discussions has been added on Page 29, Lines 485-487.

\textbf{Comment 5.} \textit{The compression block problem really comes from large-deformation formulations. For a linear problem is does not make sense to do four levels of loading because the displacement results will just linearly scale with the applied load. So, just one load level is enough, unless the authors are willing to develop a large-deformation formulation... From the results it's not clear if the formulation is converging because no comparison is made to anything. Perhaps a computation using B-Bar or Taylor-Hood can be shown for comparison?}

\textbf{Response:} 
Thanks for the review.
Following the reviewer's suggestion, the convergence study of the compression block problem has been revised back to the original one--load--level setting in Figure 30 on Page 37 of the revised manuscript.
Moreover, the convergence results using Tet4--MINI and H8P1 elements have been added as comparison methods.
The results show that the proposed mixed formulations using Tet4--RK and Hex8--RK elements achieve comparable convergence properties to the Tet4--MINI and H8P1 elements.

\section*{Reviewer \#2}

\textit{Thank you for your detailed responses.}

\textit{While your revisions address all the raised points, several key clarifications remain only in your response letter and are missing from the manuscript itself.}

\textit{Please add the following directly into the text:}

\textbf{Comment 1.} \textit{ Table 1: Condition number and efficiency comparison for Cook's membrane problem}

\textbf{Response:}
Thanks for the reviewer's suggestion. The condition number and efficiency comparison for Cook's membrane problem have been added in Table 5 on Page 32 of the revised manuscript, 
additionally, the related discussions have also been added in Page 36, Lines 520-527 of the revised manuscript.

\textbf{Comment 2.} \textit{ In the abstract, it should be mentioned that you test only structured meshes}

\textbf{Response:} 
Thanks for reviewer's suggestion.
However, to clarify this point, we have added a strain and pressure error sensitivity study with respect to $n_p$ using unstructured discretization.
The results are presented in Figure 15 on Page 26 of the revised manuscript.
This new study demonstrates that the proposed optimal constraint ratio is also effective with unstructured discretizations.
The discussion of the results for unstructured mesh is presented on Page 25, Lines 443-452.

\textbf{Comment 3.} \textit{In the conclusions, it should be mentioned that this method has worse conditioning than mesh-based methods (as many mesh-free methods do)}

\textbf{Response:}
Thanks, following the reviewer's suggestion,
the related statements for worse condition number and efficiency of the proposed mixed formulations have been added in conclusions on Page 20:
\begin{quote}
However, the implicit expression of shape functions and the larger bandwidth of the stiffness matrix in meshfree approximation also lead to larger condition number and lower efficiency compared with traditional FE formulations.
\end{quote} 

\textit{I look forward to your updated manuscript.}

\textbf{Response:} 
Authors are very thankful for the reviewer's positive and constructive comments.
Your suggestions have been invaluable in helping us improve this manuscript.

% \bibliography{references.bib}
% \begin{enumerate}[{[1]}]
% \end{enumerate}
\end{document}
