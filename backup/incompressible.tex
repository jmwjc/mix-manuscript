\section{Mixed and penalty formulations for nearly-incompressible elasticity problems}
\subsection{Penalty formulation}
Consider a body $\Omega \in \mathbb R^{n_d}$ with boundary $\Gamma$ in $n_d$-dimension, where the $\Gamma_t$ and $\Gamma_g$ denotes its natural boundary and essential boundary such that $\Gamma_t \cup \Gamma_g = \Gamma$, $\Gamma_t \cap \Gamma_g = \varnothing$. The corresponding governing equations are given by:
\begin{equation}\label{strong_penalty}
\begin{cases}
    \nabla \cdot \boldsymbol \sigma + \boldsymbol b = \boldsymbol 0 & \mathrm{in} \; \Omega \\
    \boldsymbol \sigma \cdot \boldsymbol n = \boldsymbol t & \mathrm{on} \; \Gamma_t \\
    \boldsymbol u = \boldsymbol g & \mathrm{on} \; \Gamma_g \\
\end{cases}
\end{equation}
in which $\boldsymbol \sigma$ denotes to stress tensor and, for isotropic linear elastic material, can be expressed by: 
\begin{equation}\label{stress_penalty}
\boldsymbol \sigma(\boldsymbol u) = 3\kappa \boldsymbol \varepsilon^v(\boldsymbol u) + 2\mu \boldsymbol \varepsilon^d(\boldsymbol u) 
\end{equation}
where $\boldsymbol \varepsilon^v$ and $\boldsymbol \varepsilon^d$ are the volumetric(dilatation) and deviatoric parts of strain tensor $\boldsymbol \varepsilon$, and these are evaluated by:
\begin{equation}
\boldsymbol \varepsilon^v(\boldsymbol u) =\frac{1}{3} \nabla \cdot \boldsymbol u \; \boldsymbol 1, \quad
\boldsymbol \varepsilon^d(\boldsymbol u) =\frac{1}{2}(\boldsymbol u \nabla + \nabla \boldsymbol u) - \boldsymbol \varepsilon^v, \quad
\boldsymbol \varepsilon^v : \boldsymbol \varepsilon^d = 0
\end{equation}
where $\boldsymbol 1 = \delta_{ij} \boldsymbol e_i \otimes \boldsymbol e_j$ is second order identity tensor.
$\kappa$, $\mu$ are the bulk modulus and shear modulus, and they can be represented by Young's modulus $E$ and Poisson's ratio $\nu$:
\begin{equation}\label{modulus_1}
\kappa = \frac{E}{2(1-2\nu)}, \quad \mu = \frac{E}{2(1+\nu)}
\end{equation}
And $\boldsymbol b$ denotes to prescribed body force in $\Omega$. $\boldsymbol t$, $\boldsymbol g$ are prescribed traction and displacement on natural and essential boundaries respectively. 

In accordance with Galerkin formulation, the displacement denoted by $\boldsymbol u$ can be got by the following weak problem: 
Find $\boldsymbol u \in V$
\begin{equation}\label{weak_penalty}
% \textrm{find} \; \boldsymbol u \in V, \\
\int_\Omega 2\mu \delta \boldsymbol \varepsilon^d : \boldsymbol \varepsilon^d d\Omega +
\int_\Omega 3\kappa \delta \boldsymbol \varepsilon^v : \boldsymbol \varepsilon^v d\Omega =
\int_{\Gamma_t} \delta \boldsymbol u \cdot \boldsymbol t d\Gamma + \int_\Omega \delta \boldsymbol u \cdot \boldsymbol b d\Omega, \quad
\forall \delta \boldsymbol u \in V
\end{equation}
where $V$ is the spaces defined by $V=\{\boldsymbol v \in H^1(\Omega)^2\;\vert\;\boldsymbol v = \boldsymbol g, \; \textrm{on} \; \Gamma_g\}$. $\delta \boldsymbol u$ is the virtual counterpart of $\boldsymbol u$, and $\delta \boldsymbol \varepsilon^v$ and $\delta \boldsymbol \varepsilon^d$ are the corresponding volumetric and deviatoric strain evaluated by $\delta \boldsymbol u$.

% TODO: check construct mesh 
In traditional finite element formulation, the entire domain $\Omega$ is discretized by a set of construct mesh with vertices $\{\boldsymbol x_I\}_{I=1}^{n_u}$ \cite{hughes2000}, where $n_u$ is the total number of vertices. Then, the displacement and its virtual counterpart can be approximated by the nodal coefficient and shape functions at $\boldsymbol x_I$'s, the approximated displacement and its virtual counterpart, namely $\boldsymbol u_h, \delta \boldsymbol u_h$ have the following form: 
\begin{equation}\label{u_h}
\boldsymbol u_h(\boldsymbol x) = \sum_{I=1}^{n_u} N_I(\boldsymbol x) \boldsymbol u_I, \quad
\delta \boldsymbol u_h(\boldsymbol x) = \sum_{I=1}^{n_u} N_I(\boldsymbol x) \delta \boldsymbol u_I
\end{equation}
where $N_I$ and $\boldsymbol u_I$ are the shape function and nodal coefficient tensor at node $\boldsymbol x_I$.
Introducing Eq. \eqref{u_h} to weak form of Eq. \eqref{weak_penalty} leads to the following Ritz-Galerkin problem:
Find $\boldsymbol u_h \in V_h$,
\begin{equation}\label{ritz_penalty}
\int_\Omega 2\mu \delta \boldsymbol \varepsilon^d_h : \boldsymbol \varepsilon^d_h d\Omega +
\int_\Omega 3\kappa \delta \boldsymbol \varepsilon^v_h : \boldsymbol \varepsilon^v_h d\Omega =
\int_{\Gamma_t} \delta \boldsymbol u_h \cdot \boldsymbol t d\Gamma + \int_\Omega \delta \boldsymbol u_h \cdot \boldsymbol b d\Omega, \quad
\forall \delta \boldsymbol u_h \in V_h
\end{equation}
where the approximate spaces $V_h \subseteq V$,
\begin{equation}
V_h = \{\boldsymbol v_h \in (\mathrm{span}\{N_I\}_{I=1}^{n_u})^2 \vert \boldsymbol v^h = \boldsymbol g,\; \mathrm{on} \; \Gamma_g\}
\end{equation}

For the arbitrariness of $\delta \boldsymbol u_h$, the above equation can be reduced by elimination of $\delta \boldsymbol u_I$'s as the following discrete equilibrium equation:
\begin{equation}\label{equilibrium_penalty}
(2\mu\boldsymbol K^d + 3\kappa\boldsymbol K^v) \boldsymbol d^u = \boldsymbol f
\end{equation}
where $\boldsymbol K^v$ and $\boldsymbol K^d$ are the volumetric and deviatoric stiffness matrices, and their components has the following forms:
\begin{equation}\label{stiffness_vol}
    \boldsymbol K^v_{IJ}= \int_{\Omega} \boldsymbol B^{vT}_I \boldsymbol B^v_J d\Omega
\end{equation}
\begin{equation}
    \boldsymbol K^d_{IJ}= \int_{\Omega} \boldsymbol B^{dT}_I \boldsymbol B^d_J d\Omega
\end{equation}
with
and $\boldsymbol f$ is the force vector and its components can be expressed by:
\begin{equation}
\boldsymbol f_I = \int_{\Gamma_t} N_I \boldsymbol t d\Gamma + \int_{\Omega} N_I \boldsymbol b d\Omega
\end{equation}
$\boldsymbol d^u$ is the coefficient vector containning $\boldsymbol u_I$'s.

% TODO: improve the expression.
It can be observed from Eq. \eqref{modulus} that, for a nearly-incompressible material, i.e. $\nu \rightarrow 0.5$, $\kappa \rightarrow \infty$. As a result, the volumetric stiffness matrix $\boldsymbol K^v$ of \eqref{stiffness_vol} services as an enforcement like penalty method to enforce the volumetric deformation to be zero, $\nabla \cdot \boldsymbol u = 0$, while the bulking modulus $\kappa$ can be regarded as a penalty parameter.
Traditional finite element formulations suffer severe volumetric locking due to this enforcement, and this is so-called volumetric locking. 
To reduce the burden of volumetric locking, the reduced the integration points in volumetric stiffness matrix. For clarity, substituting numerical integration to the volumetric part of weak form in Eq. \eqref{ritz_penalty} leads to:
\begin{equation}
\int_\Omega 3\kappa \delta \boldsymbol \varepsilon^v_h : \boldsymbol \varepsilon^v_h d\Omega \approx
\sum_{C=1}^{n_e}\sum_{G=1}^{n_g} 3\kappa \nabla \cdot \delta \boldsymbol u_h(\boldsymbol x_G) \nabla \cdot \boldsymbol u_h(\boldsymbol x_G) w_G
\end{equation}
The corresponding components of volumetric stiffness $\boldsymbol K^v$ in Eq. \eqref{stiffness_vol} yields:
\begin{equation}
    \boldsymbol K^v_{IJ} \approx \bar{\boldsymbol K}^v_{IJ} = \sum_{C=1}^{n_e}\sum_{G=1}^{n_g} \boldsymbol B^{vT}_I(\boldsymbol x_G) \boldsymbol B_J^v(\boldsymbol x_G) w_G
\end{equation}
where $\boldsymbol x_G$'s and $w_G$'s are the positions and weights of integration points. $n_g$ is the total number of integration points in each element, thus the total integration point is $n_c \times n_g$. The reduced integration formulations use less number of integration points compared with traditional full integration scheme. For instance, the conventional quadrilateral element use $2\times 2$ Gauss integration points as full integration, the full integration means that the stiffness matrix is exactly evaluated by this integration scheme. And for reduced integration formulation, the number of integration points is reduced from 4 to 1.

\subsection{Mixed formulation}
Another approach to alleviate the volumetric locking is using the mixed-formulation. In this approach, the pressure is approximated by another way as follows:
\begin{equation}\label{stress_mix}
\boldsymbol \sigma(\boldsymbol u, p) = p \boldsymbol 1 + 2\mu \boldsymbol \varepsilon^d(\boldsymbol u)
\end{equation}
The strong form for mixed-formulation can be rephrased as:
\begin{equation}\label{strong_mix}
\begin{cases}
    \nabla \cdot \boldsymbol \sigma + \boldsymbol b = \boldsymbol 0 & \mathrm{in} \; \Omega \\
    \frac{p}{\kappa} + \nabla \cdot \boldsymbol u = 0 & \mathrm{in} \; \Omega \\
    \boldsymbol \sigma \cdot \boldsymbol n = \boldsymbol t & \mathrm{on} \; \Gamma_t \\
    \boldsymbol u = \boldsymbol g & \mathrm{on} \; \Gamma_g \\
\end{cases}
\end{equation}
where $p\in Q$, $Q = \{q \in L^2(\Omega) \vert \int_{\Omega} q d\Omega = 0\}$.

In traditional mixed formulations, the pressure $p$ are discretized by different sets of controlled nodes, namely displacement nodes $\{\boldsymbol x_I\}_{I=1}^{n_d}$ and pressure nodes $\{\boldsymbol x_K\}_{K=1}^{n_p}$, where $n_d$ and $n_p$ are the total number of displacement nodes and pressure nodes. And then the approximate displacement denoted by $\boldsymbol u_h$ and approximate pressure denoted by $p_h$ can be expressed by  
\begin{equation}\label{p_h}
    p_h(\boldsymbol x) = \sum_{K=1}^{n_p} \Psi_K(\boldsymbol x) p_K, \quad
    \delta p_h(\boldsymbol x) = \sum_{K=1}^{n_p} \Psi_K(\boldsymbol x) \delta p_K
\end{equation}
where $p_K$'s are the coefficients. and $N_I^d$, $N_K^p$ are the corresponding shape functions.
the corresponding Ritz-Galerkin problem is that:
Find $\boldsymbol u_h \in V_h$, $p_h \in Q_h$
\begin{subequations}\label{ritz_mix}
\begin{alignat}{2}
\label{ritz_mix_1}
\int_\Omega 2\mu \delta \boldsymbol \varepsilon^d_h : \boldsymbol \varepsilon^d_h d\Omega +
\int_\Omega \nabla \cdot \delta \boldsymbol u_h p_h d\Omega &=
\int_{\Gamma_t} \delta \boldsymbol u_h \cdot \boldsymbol t d\Gamma + \int_\Omega \delta \boldsymbol u_h \cdot \boldsymbol b d\Omega, \quad
&\forall \delta \boldsymbol u_h \in V_h \\
\label{ritz_mix_2}
\int_\Omega \delta p_h \nabla \cdot \boldsymbol u_h d\Omega - \int_\Omega \frac{1}{3\kappa} \delta p_h p_h d\Omega &= 0, &\forall \delta p_h \in Q_h
\end{alignat}
\end{subequations}
where $Q_h \subseteq Q$ are defined by:
\begin{equation}
Q_h = \{q_h \in \mathrm{span}\{\Psi_K\}_{K=1}^{n_p} \vert \int_{\Omega} q_h d\Omega = 0\}
\end{equation}

With the arbitrariness of $\boldsymbol v_h$ and $q_h$, the Eq.\eqref{} leads to the following discrete governing equations:
\begin{equation}\label{equilibrium_mix}
    \begin{bmatrix}
        2\mu\boldsymbol K^{uu} & \boldsymbol K^{up} \\ (\boldsymbol K^{up})^T & -\frac{1}{3\kappa}\boldsymbol K^{pp}
    \end{bmatrix}
    \begin{Bmatrix}
        \boldsymbol d^u \\ \boldsymbol d^p 
    \end{Bmatrix} =
    \begin{Bmatrix}
        \boldsymbol f \\ \boldsymbol 0 
    \end{Bmatrix}
\end{equation}
where $\boldsymbol K^{uu} = \boldsymbol K^d$. 

From the second equation of governing equilibrium equations in Eq. \eqref{equilibrium_mix}, the coefficient vector $\boldsymbol d^p$ can be expressed by $\boldsymbol d^u$ as follows:
\begin{equation}
    \boldsymbol d^p = 3\kappa(\boldsymbol K^{pp})^{-1} (\boldsymbol K^{up})^T \boldsymbol d^u
\end{equation}
Further substituting the above equation into first equation of Eq. \eqref{equilibrium_mix} leads to:
\begin{equation}\label{equilibrium_projection}
\begin{split}
    &(2\mu\underbrace{\boldsymbol K^{uu}}_{\boldsymbol K^d} + 3\kappa \underbrace{\boldsymbol K^{up}(\boldsymbol K^{pp})^{-1}(\boldsymbol K^{up})^{T}}_{\tilde{\boldsymbol K}^v}) \boldsymbol d^u = \boldsymbol f \\
    \Rightarrow\;& (2\mu \boldsymbol K^d + 3\kappa \tilde{\boldsymbol K}^v) = \boldsymbol f
\end{split}
\end{equation}

\subsection{Equivalence between penalty-- and mixed--formulation}
It can be observed from the weak form for mixed-formulation in Eq. \eqref{ritz_mix_2} or the discrete equation shown in Eq. \eqref{equilibrium_projection} that, the solution of pressure $p_h$ is an orthogonal projection of $3\kappa \nabla \cdot \boldsymbol u_h$. Let $P_h: V_h \rightarrow P_h(V_h)$ such that $P_h(V_h) \subseteq Q_h$, where $P_h(V_h) = \textrm{Im}\:P_h$ is the image of linear operator $P_h$ \cite{philippeg.2013}. Under this circumstance, $p_h = P_h (3\kappa \nabla \cdot \boldsymbol u_h) = 3\kappa \tilde \nabla \cdot \boldsymbol u_h$, and the Eq. \eqref{ritz_mix_2} can be rephrased as: 
\begin{equation}
    \int_\Omega q_h(\nabla \cdot \boldsymbol u_h - \tilde \nabla \cdot \boldsymbol u_h) d\Omega = 0, \quad \forall q_h \in Q_h
\end{equation}
Accordingly, the corresponding volumetric part of weak form turns to:
\begin{equation}\label{projection_mixed}
\begin{split}
    \int_\Omega \nabla \cdot \delta \boldsymbol u_h p_h d\Omega &= \underbrace{\int_\Omega (\nabla \cdot \boldsymbol u_h - \tilde \nabla \cdot \delta \boldsymbol u_h) p_h d\Omega}_0 + \int_\Omega \tilde \nabla \cdot \delta \boldsymbol u_h \underbrace{p_h}_{\tilde \nabla \cdot \boldsymbol u_h} d\Omega \\
    &= \int_\Omega 3\kappa \tilde \nabla \cdot \delta \boldsymbol u_h \tilde \nabla \cdot \boldsymbol u_h d\Omega \\
\end{split}
\end{equation}
and the Ritz-Galerkin formulation becomes to: Find $\boldsymbol u_h \in V_h$
\begin{equation}
\int_\Omega 2\mu \delta \boldsymbol \varepsilon^d_h : \boldsymbol \varepsilon^d_h d\Omega +
\int_\Omega 3\kappa \tilde \nabla \cdot \delta \boldsymbol u_h \tilde \nabla \cdot \boldsymbol u_h d\Omega =
\int_{\Gamma_t} \delta \boldsymbol u_h \cdot \boldsymbol t d\Gamma + \int_\Omega \delta \boldsymbol u_h \cdot \boldsymbol b d\Omega, \quad \forall \boldsymbol u_h \in V_h
\end{equation}

In contrast, for penalty formulation, the reduced numerical integration also can be regarded as a projection. Let $\varrho_i$ be the orthogonal polynomials, 
\begin{equation}
\int_{\Omega_C} \varrho_i \varrho_j d\Omega = \begin{cases}
    J_C w_i  & i = j \\
    0 & i \ne j
\end{cases}
\end{equation}
The orthogonal interpolation $T^{k}: V \rightarrow W^{k}$, where $W^{k}$ is the interpolation space spanned by $k$ orthogonal polynomials:
\begin{equation}
    W^{k}:= \mathrm{span}\{\varrho_i \}_{i=1}^{k}
\end{equation}
For traditional Gauss-type integration scheme, $\varrho_i(\boldsymbol x_j) = \delta_{ij}$, $\boldsymbol x_j$'s are the positions of integration points. The volumetric strain can be depicted by orthogonal interpolation as:
\begin{equation}
    \nabla \cdot \boldsymbol u_h(\boldsymbol x) \approx \bar \nabla \cdot \boldsymbol u_h(\boldsymbol x) = \sum_{G=1}^{n_g} \varrho_G(\boldsymbol x) \nabla \cdot \boldsymbol u_h(\boldsymbol x_G), \quad \nabla \cdot \boldsymbol u_h(\boldsymbol x_G) = \bar \nabla \cdot \boldsymbol u_h(\boldsymbol x_G)
\end{equation}
while the integration points are regarded as interpolation coefficients. While the total number of integration points $n_g$ is lower than full integration, these means $\nabla \cdot \boldsymbol u_h$ projects to a subspace.
\begin{equation}\label{projection_penalty}
\begin{split}
\int_\Omega 3\kappa \bar \nabla \cdot \delta \boldsymbol u_h \bar \nabla \cdot \boldsymbol u_h d\Omega
    &= \sum_{C=1}^{n_e} \sum_{G,L=1}^{n_g} 3\kappa \nabla \cdot \delta \boldsymbol u_h(\boldsymbol x_G) \nabla \cdot \boldsymbol u_h(\boldsymbol x_L) \int_\Omega \varrho_G \varrho_L d\Omega  \\
    &= \sum_{C=1}^{n_e} \sum_{G=1}^{n_g} 3\kappa \nabla \cdot \delta \boldsymbol u_h(\boldsymbol x_G) \nabla \cdot \boldsymbol u_h(\boldsymbol x_G) J_C w_G \\
\end{split}
\end{equation}

With comparison of Eqs. \eqref{projection_penalty} and \eqref{projection_mixed}, the penalty formulation is actually equivalence with mixed formulation that, all approaches can be described by projection format.
