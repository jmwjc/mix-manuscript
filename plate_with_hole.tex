\subsection{Plate with hole problem}
Consider an infinite plate with a hole centered at the origin, as shown in Figure \ref{fg:plate_with_hole_model},
and at the infinity towards $x$--direction subjected an uniform traction $T=1000$.
The geometric and material parameters for this problem is that the ratio of the hole $a=1$, Young's modulus $E=3\times 10^6$ and Poisson's ratio $\nu = 0.5-10^{-8}$.
The analytical solution of this problem refers the Michell solution \cite{timoshenko1969theory} as:
\begin{equation}\label{plate_with_hole_exact}
    \left \{
    \begin{aligned}
        u_x(\rho,\theta)&=\frac{Ta}{8\mu}\left (\frac{\rho}{a}(k+1)\cos\theta-\frac{2a^3}{\rho^3}\cos3\theta    +\frac{2a}{\rho}((1+k)\cos\theta+\cos3\theta) \right )\\
        u_y(\rho,\theta)&=\frac{Ta}{8\mu}\left (\frac{\rho}{a}(k-3)\sin\theta-\frac{2a^3}{\rho^3}\sin3\theta    +\frac{2a}{\rho}((1-k)\sin\theta+\sin3\theta) \right )  
    \end{aligned}
    \right .
\end{equation}
in which $k = \frac{3-\nu}{1+\nu}$, $\mu = \frac{E}{2(1+\nu)}$.
And the stress components are given by:
\begin{equation}
    \left \{
    \begin{aligned}
        \sigma_{xx}&=T\left (1-\frac{a^2}{\rho^2}(\frac{3}{2}\cos2\theta+\cos4\theta)+\frac{3a^4}{2\rho^4}\cos4\theta\right )\\
        \sigma_{yy}&=-T\left (\frac{a^2}{\rho^2}(\frac{1}{2}\cos2\theta-\cos4\theta)+\frac{3a^4}{2\rho^4}\cos4\theta\right )\\
        \sigma_{xy}&=-T\left (\frac{a^2}{\rho^2}(\frac{1}{2}\sin2\theta+\sin4\theta)-\frac{3a^4}{2\rho^4}\sin4\theta\right )\\
    \end{aligned}
    \right .
\end{equation}

According to the symmetry property of this problem, only quarter model with length $b=5$ is considered as shown in Figure \ref{fg:plate_with_hole_model}.
The displacement is discretized by 3--node and 6--node triangular elements with $81$, $299$, $1089$ and $4225$ nodes. 
The corresponding linear and quadratic meshfree formulations are employed for pressure discretization, and the characterized support sizes are chosen as 1.5 and 2.5 respectively.
Figure \ref{fg:plate_with_hole_ns} studies the relationship between strain, pressure errors and $n_p$,
unlike the quadrilateral element case in Section \ref{sec:cantilever},
the quadaratic Tri6--RK shows worse results while the constraint ratio out of the optimal range.
And Tri3--RK exhibits less sensitivity in strain error than Tri6--RK, but its error is increasing while the $n_p$ goes up.
Both Tri3--RK and Tri6--RK with constraint ratio under optimal range performance an acceptable result.
The corresponding error convergence study is presented in Figure \ref{fg:plate_with_hole_convergece},
only Tri3--RK with $r=2$ shows a comparable result with the optimal one with $r=r_{opt}$,
the other formulations with traditional constraint ratio show lower accuracy and error convergence rate.


\begin{figure}[H]
\centering
\includegraphics[width=\textwidth]{png/plate_with_hole_model.png}
\caption{Illustration of plate with hole problem}\label{fg:plate_with_hole_model}
\end{figure}

\begin{figure}[H]
\centering
\begin{tabular}{c@{\hspace{0pt}}c}
    $\Vert \boldsymbol u - \boldsymbol u_h \Vert_V$ & $\Vert p - p_h \Vert_Q$ \\
    \raisebox{-0.7\height}{\includegraphics[width=0.48\textwidth]{png/plate_Hdev_4.png}}
    & \raisebox{-0.7\height}{\includegraphics[width=0.48\textwidth]{png/plate_L2_p_4.png}}
    \\
    \raisebox{-0.7\height}{\includegraphics[width=0.48\textwidth]{png/plate_Hdev_8.png}}
    & \raisebox{-0.7\height}{\includegraphics[width=0.48\textwidth]{png/plate_L2_p_8.png}}
    \\
    \raisebox{-0.7\height}{\includegraphics[width=0.48\textwidth]{png/plate_Hdev_16.png}}
    & \raisebox{-0.7\height}{\includegraphics[width=0.48\textwidth]{png/plate_L2_p_16.png}}
    \\
    \raisebox{-0.7\height}{\includegraphics[width=0.48\textwidth]{png/plate_Hdev_32.png}}
    & \raisebox{-0.7\height}{\includegraphics[width=0.48\textwidth]{png/plate_L2_p_32.png}}
    \\
\end{tabular}
\caption{Strain and pressures errors v.s. $n_p$ for plate with hole problem}\label{fg:plate_with_hole_ns}
\end{figure}

\begin{figure}[H]
\centering
\includegraphics[width=0.49\textwidth]{png/plate_with_hole_Hdev.png}
\includegraphics[width=0.49\textwidth]{png/plate_with_hole_L2_p.png}
\caption{Error convergence study for plate with a hole problem: (a) Strain, (b) Pressure}\label{fg:plate_with_hole_convergece}
\end{figure}

