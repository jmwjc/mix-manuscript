\section{Mixed FE--meshfree formulation with optimal constraint ratio}

In the proposed mixed--formulation, the displacement is approximated using 3--node(Tri3), 6--node(Tri6) triangular elements and 4--node(Quad4), 8-node(Quad8) quadrilateral elements in 2D, 4--node(Tet4) tetrahedral element and 8--node(Hex8) hexahedral element in 3D \cite{hughes2000}. In order to flexibly adjust to let the DOFs of pressure meet the optimal constraint, the reproducing kernel meshfree approximation is involved to approximate pressure.

\subsection{Reproducing kernel meshfree approximation}

In accordance with the reproducing kernel approximation, the entire domain $\Omega$, as shown in Figure \ref{fg:rk_approximation}, is discretized by $n_p$ meshfree nodes, $\{\boldsymbol{x}_I\}_{I=1}^{n_p}$. The approximated pressure, namely $p_h$, can be expressed by the shape function $\Psi_I$ and nodal coefficient $p_I$, yields:
\begin{equation}
p_h(\boldsymbol{x}) = \sum_{I=1}^{n_p} \Psi_I(\boldsymbol{x}) p_I
\end{equation}
where, in the reproducing kernel approximation framework, the shape function $\Psi_I$ is given by:
\begin{equation}\label{rkshape}
\Psi_I(\boldsymbol{x}) = \boldsymbol{c}(\boldsymbol{x}_I-\boldsymbol{x}) \boldsymbol{p}(\boldsymbol{x}_I-\boldsymbol{x}) \phi(\boldsymbol{x}_I - \boldsymbol{x})
\end{equation}
in which $\boldsymbol{p}$ is the basis vector, for instance in the context of the 3D quadratic case, the basis vector takes the following form:
\begin{equation}
\boldsymbol{p}(\boldsymbol{x}) = \{ 1, x, y, z, x^2, y^2, z^2, xy, xz, yz\}^T
\end{equation}
and $\phi$ stands for the kernel function. In this work, the traditional Cubic B--spline function with square or cube support is used as the kernel function:
\begin{equation}
\phi(\boldsymbol{x}_I-\boldsymbol{x}) = \phi(s_x) \phi(s_y) \phi(s_z), \quad s_i = \frac{\|\boldsymbol{x}_I - \boldsymbol{x}\|}{\bar{s}_{iI}}
\end{equation}
with
\begin{equation}
\phi(s) = \frac{1}{3!} \begin{cases}
(2-2s)^3 - 4(1-2s)^3 & s\le\frac{1}{2} \\
(2-2s)^3 &\frac{1}{2}<s<1 \\
0 & s> 1
\end{cases}
\end{equation}
where $\bar{s}_{iI}$'s are the support size towards the $i$-direction for the shape function $\Psi_I$. The correction function $\boldsymbol{c}$ can be determined by the following so-called consistency condition:
\begin{equation}\label{cc1}
\sum_{I=1}^{n_p} \Psi_I(\boldsymbol{x}) \boldsymbol{p}(\boldsymbol{x}_I) = \boldsymbol{p} (\boldsymbol{x})
\end{equation}
or equivalent shifted form:
\begin{equation}\label{cc2}
\sum_{I=1}^{n_p} \Psi_I(\boldsymbol{x}) \boldsymbol{p}(\boldsymbol{x}_I-\boldsymbol{x}) = \boldsymbol{p} (\boldsymbol{0})
\end{equation}
Substituting Eq. \ref{rkshape} into Eq. \eqref{cc2} leads to:
\begin{equation}\label{correction}
\boldsymbol{c}(\boldsymbol{x}_I-\boldsymbol{x}) = \boldsymbol{A}^{-1}(\boldsymbol{x}_I-\boldsymbol{x}) \boldsymbol{p}(\boldsymbol{0})
\end{equation}
in which $\boldsymbol{A}$ is namely the moment matrix evaluated by:
\begin{equation}
\boldsymbol{A}(\boldsymbol{x}_I-\boldsymbol{x}) = \sum_{I=1}^{n_p} \boldsymbol{p}(\boldsymbol{x}_I-\boldsymbol{x}) \boldsymbol{p}^T(\boldsymbol{x}_I-\boldsymbol{x}) \phi(\boldsymbol{x}_I-\boldsymbol{x})
\end{equation}
Taking Eq. \eqref{correction} back to Eq. \eqref{rkshape}, the final form of the reproducing kernel shape function can be obtained as:
\begin{equation}
\Psi_I(\boldsymbol{x}) = \boldsymbol{p}^T(\boldsymbol{0}) \boldsymbol{A}^{-1}(\boldsymbol{x}_I-\boldsymbol{x}) \phi(\boldsymbol{x}_I-\boldsymbol{x})
\end{equation}

As shown in Figure \ref{fg:rk_approximation},
reproducing kernel meshfree shape functions are globally smooth across the entire domain,
using them to discretize the pressure field allows the constraint ratio to be adjusted arbitrarily, without being limited by element topology.
Moreover, when combined with finite element approximations in Eq. \ref{ritz_Galerkin},
numerical integration can be conveniently performed within each finite element ($\Omega_C's$).

\begin{figure}[H]
\centering
\includegraphics[width=\textwidth]{png/mix.png}
\caption{Illustration for reproducing kernel meshfree approximation}\label{fg:rk_approximation}
\end{figure}

\subsection{Pressure node distributions with optimal constraint ratio}

In this subsection, the 2D and 3D inf--sup tests \cite{chapelle1993} with the mixed FE-meshfree formulations are employed to validate the proposed estimator of the inf-sup value. The 2D test considers the square domain $\Omega = (0,1)\otimes (0,1)$ in Figure \ref{fg:infsup_convergence_2D}, where the displacement is discretized by Tri3 element, Quad4 element with $4\times 4$, $8\times 8$, $16\times 16$ and $32\times 32$ elements, Tri6 element, Quad8 element with $2\times 2$, $4\times 4$, $8\times 8$ and $16\times 16$ elements, respectively. The 3D test employs a cube domain $\Omega = (0,1)\otimes (0,1)\otimes (0,1)$ in Figure \ref{fg:infsup_convergence_3D} with $4\times 4$, $8\times 8$ and $16\times 16$ elements for the Tet4 element and Hex8 element. In order to avoid the influence of interpolation error, uniform nodal distributions are used for pressure discretizations.

\begin{figure}[H]
\centering
\begin{subcaptiongroup}
\parbox[b]{0.4\textwidth}{
\includegraphics[width=0.38\textwidth]{png/infsup_model.png}\caption{2D test model}\label{fg:infsup_convergence_2D}
}
\hspace{20pt}
\parbox[b]{0.4\textwidth}{
\includegraphics[width=0.4\textwidth]{png/inf_sup_block.png}\caption{3D test model}\label{fg:infsup_convergence_3D}
}
\end{subcaptiongroup}
\caption{Illustration of inf--sup test}\label{fg:inf_sup_test}
\end{figure}

Figures \ref{fg:infsup_convergence_2D_a}--\ref{fg:infsup_convergence_3D_b} show the corresponding results, in which the red line stands for the value of $\beta$ with respect to the number of pressure nodes $n_p$, and the vertical dashed line denotes the stabilized number $n_s$. The deeper color of the lines means mesh refinement. The results show that, no matter linear or quadratic elements, as $n_p$ increases over $n_s$, the value of $\beta$ sharply decreases, and then the inf-sup condition cannot be maintained. This result is consistent with the discussion in Section \ref{sec:constraint_ratio}, and again verifies the effect of the proposed estimator.

\begin{figure}[H]
\centering
\includegraphics[width=\textwidth]{png/tri3.png}
\caption{Inf--sup test for Tri3--RK}\label{fg:infsup_convergence_2D_a}
\end{figure}

\begin{figure}[H]
\centering
\includegraphics[width=\textwidth]{png/tri6.png}\caption{Inf--sup test for Tri6--RK}\label{fg:infsup_convergence_2D_b}
\end{figure}

\begin{figure}[H]
\centering
\includegraphics[width=\textwidth]{png/quad4.png}\caption{Inf--sup test for Quad4--RK}\label{fg:infsup_convergence_2D_c}
\end{figure}

\begin{figure}[H]
\centering
\includegraphics[width=\textwidth]{png/quad8.png}\caption{Inf--sup test for Quad8--RK}\label{fg:infsup_convergence_2D_d}
\end{figure}

\begin{figure}[H]
\centering
\includegraphics[width=\textwidth]{png/Tet4.png}\caption{Inf--sup test for Tet4--RK}\label{fg:infsup_convergence_3D_a}
\end{figure}

\begin{figure}[H]
\centering
\includegraphics[width=\textwidth]{png/Hex8.png}\caption{Inf--sup test for Hex8--RK}\label{fg:infsup_convergence_3D_b}
\end{figure}

Moreover, the mixed formulation's results with the traditional optimal constraint ratio $r=n_d$ are listed in these figures as well, and $\beta$ in this circumstance is already much smaller than those in the optimal range. Considering the results shown above, the easy programming and efficiency, the pressure nodes are chosen among the displacement nodes. The final schemes for linear and quadratic, 2D and 3D element discretizations are shown in Figure \ref{fg:mix_scheme}, in which all constraint ratios belong to the range of the optimal ratio. The corresponding inf-sup test results for these schemes are also marked in inf--sup test figure and show that, with mesh refinement, their $\beta$'s are always maintained at a non-negligible level.

\begin{figure}[htb]
\centering
\begin{tabular}{c@{\hspace{24pt}}c}
\includegraphics[width=0.3\textwidth]{png/mix_tri3.png} &
\includegraphics[width=0.3\textwidth]{png/mix_tri6.png} \\
Tri3--RK & Tri6--RK \\
\includegraphics[width=0.3\textwidth]{png/mix_quad4.png} &
\includegraphics[width=0.3\textwidth]{png/mix_quad8.png} \\
Quad4--RK & Quad8--RK \\
\includegraphics[width=0.3\textwidth]{png/mix_tet4.png} &
\includegraphics[width=0.3\textwidth]{png/mix_hex8.png} \\
Tet4--RK & Hex8--RK \\
\raisebox{-0.3\height}{\includegraphics[width=12pt]{png/legend_u.png}} :Displacement node &
\raisebox{-0.3\height}{\includegraphics[width=12pt]{png/legend_p.png}} :Pressure node
\end{tabular}
\caption{Nodal distribution schemes for mixed FE-meshfree formulations with $r = r_{opt}$}\label{fg:mix_scheme}
\end{figure}

