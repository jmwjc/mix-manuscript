\section{FEM--meshfree mixed--formulation with optimal constraint}

\subsection{Finite element and reproducing kernel approximations}

In the proposed mixed-formulation, the displacement is approximated using three-node, six-node triangular elements and four-node, eight-node quadrilateral elements \cite{hughes2000}. In order to flexcially adjust to let the dofs of pressure meets to be optimal, the reproducing kernel meshfree approximation is involved to approximate pressure.

In accordance with the reproducing kernel approximation, the entire domain $\Omega$ is discretized by $n_p$ meshfree points, $\{\boldsymbol x_I\}_{I=1}^{n_p}$. Each meshfree point equips a meshfree shape function $\Psi_I$ and nodal coefficient $p_I$, and the approximated pressure namely $p_h$ can be presented by:
\begin{equation}
p_h(\boldsymbol x) = \sum_{I=1}^{n_p} \Psi_I(\boldsymbol x) p_I
\end{equation}
where, in the reproducing kernel approximation framework, the shape function $\Psi_I$ is given by:
\begin{equation}\label{rkshape}
\Psi_I(\boldsymbol x) = \boldsymbol c(\boldsymbol x_I-\boldsymbol x) \boldsymbol p(\boldsymbol x_I-\boldsymbol x) \phi(\boldsymbol x_I - \boldsymbol x)
\end{equation}
in which $\boldsymbol p$ is the basis function, especially for 2D quadratic basis function, having the following form: 
\begin{equation}
\boldsymbol p(\boldsymbol x) = \{ 1, x, y, x^2, xy, y^2\}^T
\end{equation}
and $\phi$ stands for the kernel function. In this work, the traditional Cubic B-spline function with square suppot is used as the kernel function:
\begin{equation}
\phi(\boldsymbol x_I-\boldsymbol x) = \phi(s_x) \phi(s_y), \quad s_i = \frac{\Vert \boldsymbol x_I - \boldsymbol x\Vert}{\bar s_{iI}}
\end{equation}
with
\begin{equation}
\phi(s) =\frac{1}{3!} \begin{cases}
    (2-2s)^3 - 4(1-2s)^3 & s\le\frac{1}{2} \\
    (2-2s)^3 &\frac{1}{2}<s<1 \\
    0 & s> 1
\end{cases}
\end{equation}
where $\bar s_{iI}$'s are the support size towards the $i$-direction for the shape function $\Psi_I$.
The correction function $\boldsymbol c$ can be determined by the following so-call consistency condition:
\begin{equation}\label{cc1}
\sum_{I=1}^{n_p}\Psi_I(\boldsymbol x) \boldsymbol p(\boldsymbol x_I) = \boldsymbol p (\boldsymbol x)
\end{equation}
or equivalent shifted form:
\begin{equation}\label{cc2}
\sum_{I=1}^{n_p}\Psi_I(\boldsymbol x) \boldsymbol p(\boldsymbol x_I-\boldsymbol x) = \boldsymbol p (\boldsymbol 0)
\end{equation}
Substituting Eq. \ref{rkshape} into Eq. \eqref{} leads to:
\begin{equation}\label{correction}
\boldsymbol c(\boldsymbol x_I-\boldsymbol x) = \boldsymbol A^{-1}(\boldsymbol x_I-\boldsymbol x)\boldsymbol p(\boldsymbol 0)
\end{equation}
in which $\boldsymbol A$ is namely moment matrix evaluating by:
\begin{equation}
\boldsymbol A(\boldsymbol x_I-\boldsymbol x) = \sum_{I=1}^{n_p}\boldsymbol p(\boldsymbol x_I-\boldsymbol x) \boldsymbol p^T(\boldsymbol x_I-\boldsymbol x)\phi(\boldsymbol x_I-\boldsymbol x)
\end{equation}
Taking Eq. \eqref{correction} back to Eq. \eqref{rkshape}, the final form of reproducing kernel shape function can be got as:
\begin{equation}
\Psi_I(\boldsymbol x) = \boldsymbol p^T(\boldsymbol 0) \boldsymbol A^{-1}(\boldsymbol x_I-\boldsymbol x)\phi(\boldsymbol x_I-\boldsymbol x)
\end{equation}

\subsection{Pressure nodes distributions}