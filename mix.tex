\section{Mixed--formulation}
\subsection{Nearly--incompressible elasticity}
Consider a body $\Omega \in \mathbb R^{n_d}$ with boundary $\Gamma$ in $n_d$-dimension, where the $\Gamma_t$ and $\Gamma_g$ denotes its natural boundary and essential boundary such that $\Gamma_t \cup \Gamma_g = \Gamma$, $\Gamma_t \cap \Gamma_g = \varnothing$. The corresponding governing equations for mixed-formulation are given by:
\begin{equation}\label{strong}
\begin{cases}
    \nabla \cdot \boldsymbol \sigma + \boldsymbol b = \boldsymbol 0 & \mathrm{in} \; \Omega \\
    \frac{p}{\kappa} + \nabla \cdot \boldsymbol u = 0 & \mathrm{in} \; \Omega \\
    \boldsymbol \sigma \cdot \boldsymbol n = \boldsymbol t & \mathrm{on} \; \Gamma_t \\
    \boldsymbol u = \boldsymbol g & \mathrm{on} \; \Gamma_g \\
\end{cases}
\end{equation}
where $\boldsymbol u$ and $p$, standing for displacement and hydrostatic pressure respectively, are the variables of this problem. $\boldsymbol \sigma$ denotes to stress tensor and has the following form: 
\begin{equation}\label{stress}
    \boldsymbol \sigma(\boldsymbol u, p) = p \boldsymbol 1 + 2\mu \nabla^d \boldsymbol u
\end{equation}
in which $\boldsymbol 1 = \delta_{ij} \boldsymbol e_i \otimes \boldsymbol e_j$ is second order identity tensor.
$\nabla^d \cdot \boldsymbol u$ is the deviatoric gradient of $\boldsymbol u$ and can be evaluated by:
\begin{equation}
\nabla^d \boldsymbol u = \frac{1}{2}(\boldsymbol u \nabla + \nabla \boldsymbol u) - \frac{1}{3} \nabla \cdot \boldsymbol u
\end{equation}
and $\kappa$, $\mu$ are the bulk modulus and shear modulus, and they can be represented by Young's modulus $E$ and Poisson's ratio $\nu$:
\begin{equation}\label{modulus}
\kappa = \frac{E}{2(1-2\nu)}, \quad \mu = \frac{E}{2(1+\nu)}
\end{equation}

Moreover, $\boldsymbol b$ denotes to prescribed body force in $\Omega$. $\boldsymbol t$, $\boldsymbol g$ are prescribed traction and displacement on natural and essential boundaries respectively. 

In accordance with Galerkin formulation, the weak form can be given by: 
Find $\boldsymbol u \in V$, $p \in Q$,
\begin{equation}\label{weak_form}
\left \{
\begin{aligned}
    a(\boldsymbol v, \boldsymbol u) + b(\boldsymbol v, p) &= f(\boldsymbol v) \quad &\forall \boldsymbol v \in V \\
    b(\boldsymbol u, q) + c(q,p) &= 0 \quad &\forall q \in Q
\end{aligned}
\right .
\end{equation}
with the spaces $V, Q$ defined by:
\begin{equation}\label{mix_formulation}
V=\{\boldsymbol v \in H^1(\Omega)^2\;\vert\;\boldsymbol v = \boldsymbol g, \; \textrm{on} \; \Gamma_g\}
\end{equation}
\begin{equation}
Q = \{q \in L^2(\Omega) \vert \int_{\Omega} q d\Omega = 0\}
\end{equation}
where $a:V\times V\rightarrow \mathbb R$, $b:V\times Q\rightarrow \mathbb R$ and $c:Q\times Q \rightarrow \mathbb R$ are bilinear forms, and $f:V\rightarrow \mathbb R$ is the linear form. In elasticity problem, they are given by:
\begin{align}
    a(\boldsymbol v, \boldsymbol u) &= \int_\Omega \nabla^d \boldsymbol v : \nabla^d \boldsymbol u d\Omega \\
    b(\boldsymbol v, q) &= \int_\Omega \nabla \cdot \boldsymbol v q d\Omega \\
    \label{form_c}
    c(q,p) &= -\int_\Omega \frac{1}{3\kappa} qpd\Omega \\
    f(\boldsymbol v) &= \int_{\Gamma_t} \boldsymbol v \cdot \boldsymbol t d\Gamma + \int_{\Omega} \boldsymbol v \cdot \boldsymbol b d\Omega
\end{align}

\subsection{Ritz--Galerkin problem and volumetric locking}
In mixed--formulation framework, the displacement and pressure can be discretized by different approximations. 
The approximant displacement $\boldsymbol u_h$ and approximant pressure $p_h$ can be expressed by:
\begin{equation}
    \boldsymbol u_h(\boldsymbol x) = \sum_{I = 1}^{n_u} N_I(\boldsymbol x) \boldsymbol u_I
    , \quad
    p_h(\boldsymbol x) = \sum_{K = 1}^{n_p} \Psi_K(\boldsymbol x) p_K
\end{equation}
leading these approximations into the weak form of Eq. \eqref{weak_form} yields the following Ritz--Galerkin problems: 
Find $\boldsymbol u_h \in V_h,\;p_h \in Q_h$,
\begin{equation}\label{ritz_Galerkin}
\left \{
\begin{aligned}
    a(\boldsymbol v_h, \boldsymbol u_h) + b(\boldsymbol v_h, p_h) &= f(\boldsymbol v_h) \quad &\forall \boldsymbol v_h \in V_h \\
    b(\boldsymbol u_h, q_h) + c(q_h,p_h) &= 0 \quad &\forall q_h \in Q_h
\end{aligned}
\right .
\end{equation}

For nearly incompressible material, the Poisson ratio approaches to $0.5$, the bulking modulus $\kappa$ will turns to be infinity based on Eq. \eqref{modulus}. 
Then the bilinear form $c$ in Eq. \eqref{form_c} turns to be zero.
And the weak form of Eq. \eqref{ritz_Galerkin} belong to an enforcement of the volumetric strain $\nabla \cdot \boldsymbol u_h$ to be zero using the Lagrangian multiplier method,
where $p_h$ is the Lagrangian multiplier.

Furthermore, from the second line of Eq. \eqref{ritz_Galerkin}, we have:
\begin{equation}
    b(\boldsymbol u_h, q_h) + c(q_h, p_h) = 
    (q_h, \nabla \cdot \boldsymbol u_h) - (q_h, \frac{1}{3\kappa} p_h) = 0
    ,\quad \forall q_h \in Q_h
\end{equation}
or
\begin{equation}\label{orthogonal}
    (q_h, 3\kappa \nabla \cdot \boldsymbol u_h - p_h) = 0
    ,\quad \forall q_h \in Q_h
\end{equation}
where $(\bullet, \bullet)$ is the inner product operator evaluated by:
\begin{equation}
    (q,p) := \int_\Omega qp d\Omega
\end{equation}
Obviously in Eq. \eqref{orthogonal},
$p_h$ is the orthogonal projection of $3\kappa \nabla \cdot \boldsymbol u_h$ regarded to the space $Q_h$ \cite{brezzi1991a},
and, for further development, use the nabla notion with upper tilde to name the projection operator, i.e. $p_h = \tilde \nabla \cdot \boldsymbol u_h$.
In this circumstance, the bilinear form $b$ in first line of Eq. \eqref{ritz_Galerkin} becomes:
\begin{equation}
    \begin{split}
        b(\boldsymbol v_h, p_h) &= (
            \underbrace{
            \nabla \cdot \boldsymbol v_h - \tilde \nabla \cdot \boldsymbol v_h, p_h
            }_{0}
        ) + (\tilde \nabla \cdot \boldsymbol v_h, \underbrace{p_h}_{3\kappa\tilde \nabla \cdot \boldsymbol u_h}) \\
        &= 
        (\tilde \nabla \cdot \boldsymbol v_h, 3\kappa\tilde \nabla \cdot \boldsymbol u_h) \\
        &= \tilde a(\boldsymbol v_h, \boldsymbol u_h)
    \end{split}
\end{equation}
where the bilinear form $\tilde a: V_h \times V_h \rightarrow \mathbb R$ is defined by:
\begin{equation}
    \tilde a(\boldsymbol v_h, \boldsymbol u_h) =
    \int_\Omega 3\kappa \tilde \nabla \cdot \boldsymbol v_h \tilde \nabla \cdot \boldsymbol u_h d\Omega
\end{equation}

Accordingly, the problem of Eq. \eqref{ritz_Galerkin} becomes to be one variable form:
Find $\boldsymbol u_h \in V_h$,
\begin{equation}\label{weak_penalty}
    a(\boldsymbol v_h, \boldsymbol u_h) + \tilde a(\boldsymbol v_h, \boldsymbol u_h) = f(\boldsymbol v_h)
    , \quad \forall \boldsymbol v_h \in V_h
\end{equation}

As $\kappa \rightarrow \infty$, Eq. \eqref{weak_penalty}can be regarded as an enforcement of volumetric strain using penalty method, 
where $\tilde a$ is the penalty term.
However, it should be noted that, if the mixed--formulation wants to get a satisfactory result, this orthogonal projection must be surjective\cite{stein2004}.
In the contrast of surjective case,
for a given $p_h\in Q_h$, it possibly cannot find a $\boldsymbol u_h \in V_h$ such that $p_h = 3\kappa \nabla \cdot \boldsymbol u_h$. 
It will lead to a much smaller displacement than expected and an oscillated pressure result. 
This phenomenon is so--call volumetric locking.

